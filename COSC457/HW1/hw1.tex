\documentclass[a4paper,12pt]{report}
\usepackage[utf8x]{inputenc}
\usepackage{url}

%opening
\title{Braid Game Review}
\author{Byron Heads \\
		E00062946}

\date{\today}
\begin{document}

\maketitle

\section*{}
The game I am reviewing the game Braid$^{[\ref{braidlink}]}$, developed by the independent game developer Jonathan Blow.  Braid is a side-scrolling puzzle game, rated E10+.  The game was first released August 6, 2008 on XBox Live Arcade.  A windows version was released on April 10, 2009, a Mac OS X version was released on May 20, 2009, release on the PlayStaion Network November 12, 2009, and a Linux version was release December 14, 2010.  Braid can be downloaded via Steam, XBox Live Arcade, the PlayStaion Network, and the websites Impulse, and Gamers Gate.  Braid can be purchased for around \$$9.99$ by any of these vendors.  I bought my version from the Humble Indie Bundle$^{[\ref{hiblink}]}$ which gave me a Steam code to download Braid and most of the other games.  The requirements for Braid are low compared to most games on the market today.

\begin{table}[hp!]
    \caption{Braid System Requirements.$^{[\ref{steamlink}]}$
    * - Linux requirements were determined from the Mac requirements.  }
    \begin{tabular}{|l|p{3cm}|p{3.5cm}|p{3cm}|}
        \hline
        System & Windows & Mac & Linux \\
        \hline
        OS & XP / Vista / 7 & Leopard, Snow Leopard & 2.6 Kernel* \\
        \hline
        CPU & 1.4GHz or better & 1.0 Ghz or better & 1GHz* \\
        \hline
        Memory & 768MB or more & 512 MB or more &  512MB*\\
        \hline
            Hard Drive & 200MB or more & 185MB or more & 185MB* \\
            \hline
            Graphics & DirectX 9.0c, Pixel Shader 2.0 & ATI Radeon\texttrademark 9500 or Better, NVIDIA GeForce\texttrademark 5900 or better, Intel GMA 950 or better & Any Supported OpenGL Driver* \\ 
            \hline
        \end{tabular}
\end{table}

Braid is a 2D side-scrolling puzzle game where the player controls the protagonist Tim.  The goal of the game is to rescue a princess from a monster.  To process through the game Tim needs to collect and assemble jigsaw puzzle pieces that are scattered throughout the game worlds.  To get the pieces Tim can control the flow of time.  Each level has a different time manipulation mechanism.  Tim can move in four directions, jump, climb, and can kill ''gumbas'' by jumping on their heads.  The player reaches the end of a level by going through a door.  At the end of each level is a flag and a castle where a dinosaur tells the player ''The prices is in another castle.''  This game posses many different concepts and messages leaving the player to interpret the game in their own way.


The story revolves are Tim and his quest to find his princess and save her from a monster.  The story is told by reading books that are unlocked at each world.  More books are unlocked by solving all of the jigsaw puzzles.  Within the story we learn Tim is trying to erase some wrong he did to the prices by rewinding time.  The story is working backwards in time back to Tim's mistake.  The different levels of the game represent different emotions that Tim has gone through.  The player's role is to unwind Tim's past and save the princess from the monster.

I installed and ran Braid on OS X Snow Leopard with Steam.  Installing the game was as simple are entering the game code I got from the Humble Indie Bundle in to Steam, clicking install, and waiting for it to download.  The game is run from Steam.  When Braid is launched, and small configuration screen is displayed where the player can adjust the games graphics setting.  Braid does not have a main menu, instead the player is sent directly to main world.  This world allows the player to access and unlocked worlds via doors, and access each worlds jigsaw puzzle.  The only interface element in Braid is to indicate how many puzzle pieces in a level are are and how many you have collected.  The player can start a new game from the options menu.

You control Tim with the arrow keys to move him around the level.  Tim can jump by pressing the space key.  Jumping on ''gumbas'' cause Tim to jump again, this also kills the ''gumba'' if time is moving forward.  Nothing can die if time is not moving forward.  You can run time backwards with the shift key.  While rewinding time you can control the speed by using the arrow keys.  The player cannot lose in Braid since you can undo your death.  There are no limitations on rewinding time, you can rewind all the way back to the start of the level.  Each world deals with time in a different way.  For example: some items are unaffected by time manipulation, in one world time only moves forward when Tim moves forward and backwards when Tim moves backwards, and another world where you get a shadow that performs your actions when you rewind time.  All of these different time manipulation levels make for many challenging and sometimes difficult puzzles.  Since the goal of the game is to unlock Tim's story and save the princess, there is no score.  Instead the player collects puzzle pieces the use to assemble different puzzles.

The artwork in Braid took over a year to create.  It was created by David Hellman, a critically-acclaimed webcomic artist$^{[\ref{artlink}]}$.  The artwork in each world is designed to express different emotions and ideas.  This way each level and world told part of the story on the game.  The graphics are all 2D and posses a cartoon style to them.  The music for Braid was created by Cheryl Ann Fulton, Shira Kammen and Jami Sieber.  The music was purposely select to create a specific mood to match the emotion of each world.  The music loops are long enough so that you do not notice that its being looped.  The most striking part of the music is that it matches the level when it is being run in reverse, sometimes hauntingly.  

The game also has a mode called speed runs.  In speed runs you have to beat six different runs in a given amount of time.  This mode gives players another challenge other then the puzzles.One of the puzzles includes beating the entire game in one hour and forty-five minutes.  The only manual for the game is player created ones you can find on gaming forums.  I was able to play Braid for three hours without any noticeable bugs or any crashes.

Braid is very entertaining and enjoyable game.  The story line is more targeted for the 16+ crowd of casual gamers.  Controlling the flow of time to solve puzzles is a fun and unique game mechanic.  The story in the game makes you want to play more.  Some of the puzzles can be frustrating ( I had to look up a few solutions online ), but overall the game is easier then may puzzle games I have played.  The ability to manipulate time and the story puts this game in the top ten puzzle games.

Braid does an good job in following requirements of good game design.  The difficult in Braid increases as the story progresses.  Each world you learn a new time manipulating ability that allows you to solve new puzzles.  Though Braid's levels and story are linear, Braid is about choice, flow, and undoing your choices.  The flow of the game is based on the idea of undoing your mistakes.  Not just the player made mistakes, but the mistakes Tim made in the main story.      

In summary, Braid is fun Indie puzzle game.  The unique story and game play sets it apart from most puzzle games.  Though some of the puzzles can be tricky to solve, they normally can be solved by good timing and thinking backwards.  Overall this is one of my favorite Indie games, with an average game rating of 9/10 from most game reviewers. 


\subsection*{Web Resources}

\begin{enumerate}
\item \label{braidlink}\url{http://braid-game.com/} Braid homepage
\item \label{hiblink}\url{http://www.humblebundle.com/} Humble Indie Bundle homepage
\item \label{artlink}\url{http://www.alessonislearned.com/} David Hellman's webcomic 
\item \label{steamlink} \url{http://store.steampowered.com/app/26800/} Braid's store page on Steam
\item \url{http://en.wikipedia.org/wiki/Braid_game} Wikipedia entry on Braid
\end{enumerate}


\end{document}

