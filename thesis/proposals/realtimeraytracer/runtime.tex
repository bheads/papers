\section{Run-Time Complexity}
Ray tracing is slower when compared to z-buffering.  Z-buffering is affected by the number of visible triangles in a scene.  Reducing the number of triangles will directly influence the performance of a z-buffering engine.  Many z-buffering engines ignore any triangles that are not in front of the camera.

In a ray tracing engine, it is possible to have a ray collision with triangles or other objects that are not directly visible in a scene.  Therefore, reducing the number of triangles has less of an impact on performance.

The runtime of the simple ray tracing algorithm [\ref{ray-trace}] can be computed as $O(p*o)$ with $p$ as the number of pixels being rendered and $o$ is the number of objects in the simulation.    The ray trace function iterates over all objects in the scene to test for a ray-object collision.  The ray trace function is called for every pixel in the output image.   The algorithm can be improved by using a different data structure to store the scene's objects and lights.  A k-d tree or octree can be used to reduce the number of collision tests and each pixel can be computed in parallel.

The runtime of the full ray tracer algorithm [\ref{ray-trace-full}] is more complex to compute.  Without recursive calls, the ray trace function has a $O( p(o + o*l ))$ with $p$ as the number of pixels, $o$ is the number of objects, and $l$ is the number of lights in the simulation.  This changes when we allow for recursion.  Each ray can cast two more rays up to a set maximum depth.  The maximum depth prevents infinite recursion and can be used yo control the algorithms overall performance and quality.  This leaves us with a runtime of $O( p( 2k( o +o*l )))$ with $k$ as the maximum depth of recursion.   Adding soft shadows and anti-aliasing can make this even more complex.

Using linear lists of objects and lights creates a serious performance hit.  Using a data structure that reduces the number of objects tested for ray-object collisions will give a dramatic performance boost to the algorithm.

In my research I will work on determining a runtime model of several different data structures.  Using this research I hope to determine a possible optimum design to achieve real-time results. 
