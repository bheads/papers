\documentclass[12pt,a4paper,oneside]{report}
\usepackage{fullpage}
\usepackage[none]{hyphenat} 

\newcommand{\HRule}{\rule{\linewidth}{0.5mm}}


\begin{document}

\begin{titlepage}
    \begin{center}
        \textsc{\large Eastern Michigan University}\\[1.5cm]
        \textsc{\large Computer Science Department}\\
        \textsc{\large Masters Thesis Proposal}\\[0.5cm]
        \HRule\\[0.4cm]
        { \huge \bfseries Cross-Platform Multi-Threaded \\
            Game Engine Development }\\[0.4cm]
        \HRule\\[1.5cm]

        % Author and supervisor
        \begin{minipage}{0.4\textwidth}
            \begin{flushleft} \large
                \emph{Author:}\\
                Byron \textsc{Heads}
            \end{flushleft}
        \end{minipage}
        \begin{minipage}{0.4\textwidth}
            \begin{flushright} \large
                \emph{Thesis Advisor:} \\
                    Dr.~William \textsc{Sverdlik}
            \end{flushright}
        \end{minipage}

        \vfill
        { \large \textbf{Abstract Summary}}\\
    
        \HRule\\[0.5cm]
        { \large \today }
    \end{center}
\end{titlepage}

\begin{abstract}
Modern computer games, simulations, and visualization software are some of the most complex and demanding software in use today.  Many of these programs need to run on inexpensive and available commercial-grade hardware.  Consumers expect games to continuously improve in visual quality, physics, and game-play complexity.  Early computer games were written as single projects that targeted specific hardware, but to continue developing this way has become to costly in resource and time.  To reduce the cost and complexity of development, game developers build there software on top of game engines.

In $2004$ the video game industry was worth \$$7.3$ billion and the industry is still growing.  Computer simulation and visualization software are used the government, military, and universities for research and other projects.  

\end{abstract}



\end{document}

