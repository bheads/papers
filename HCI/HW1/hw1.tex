\documentclass[a4paper,10pt]{article}
\usepackage[utf8x]{inputenc}

%opening
\title{Small Business Storage Management and Gate Control System}
\author{Byron Heads \\
		E00062946}

\begin{document}

\maketitle
%\begin{abstract}
%Project parts 1, 2, and 3.
%\end{abstract}

\paragraph*{}
A modern storage facility can have several hundred clients.  To track client information, payments,
and to give secure and convent access to their storage requires several key software components.
This system will have a site manager, and customer as its primary user base.  The storage facility's
customer, known as the renter, will have access to the Gate Input Interface.  The site manager will 
will be access the Renter Database, Payment System, and Gate Control System.  These systems maybe
accessed separately, but are all integrated in the main programs database.

The Gate Input Interface consists of two identical keypads with a small LCD screen, and the main 
access gate.  A keypad is located on each side of the gate.  The bottom of the keypad is 41-45 inches
from the road surface, this the same height as residential mailboxes.  The distance from the gate
is roughly 5-8 feet.  The renter pulls up next to the keypad and enters a seven digit code.  The code
is displayed on screen as its typed.  The screen is automatically cleared after two minutes of inactivity.
After the renter enters their code the press the accept key.  The screen will either display a welcome
message followed by the gate opening, or the screen displays "Invalid Code".  Five invalid codes cause
the keypad to lockout for two minutes to discourage code guessing.  Renters that have not paid their
rent are locked up, and the screen displays "See Manager". All code entries are logged in the Gate 
Control System database.

The Renter Database is where the site manager will spend most of their time using.  The database is tracks
all renter information.  This includes renter contact information, units rented, payment due dates, rental
rates, and any other collected notes on the renter.  The manager can search with the renter search screen.
This screen allows the manager to search by name, unit numbers, or a generic search.  The screen will
return a list of search results with links to the resulting renter record.

Payments are applied through the Payment System, and can be accessed via two different methods.  First
is in the Payment System.  The manager selects apply payment, then uses the renter search screen to 
select the user to apply a payment to.  If a valid renter is selected, the manager is taken to the payment
screen.  The manager then selects the payment method, amount, and date to issue the payment.  The system
uses the last payment method that the renter used.  The payment amount is calculated from the the issue
payment date using the renters agreed rate.  This gives the manager the ability to prorate payments.
Payments are applied to the renters account, and their paid up to date is adjusted.  If the renter had
been locked out and is now paid up, the lockout will be removed.  The Payment System is also used to
produce reports on renters that are behind on payments, or have a payment coming out.  The global 
settings for renter lockout is adjusted at this screen.  Lockout rules can be set for individual renters
under the Renter Database screen's advanced tab.
 
The Gate Control System is used to set the gate settings.  Most gate settings are global and can be
adjusted per user in the Renter Database advanced tab.  From this system the manager can control the
gate access times, and the gate access log.  The time setting is used to limit access to the facility 
while its closed.  The gate log displays all user access to the keypads.    

Within the storage management system there are two users.  The renter, and the manager.  Both users
have very different usability domains and goals.  The renter is the managers customer, and is assumed
to have little to no experience with using computers, or security systems.  To deal with this the keypad 
interface was modeled after a telephone keypad.  The use of this common interface gives the renter a 
familiar interface that most renters should have experience and knowledge with using.  The purpose of 
the interface is to give the renter access to the facility without risking site security.  Speed of
access to the site will also be important to the renter.     

The manager will be a moderate to advanced user.  The manager will be using the system every day, and
for all transactions.  The manager is expected to be an expert in storage facility management.  This 
includes tracking renters, payments, security, and customer service.  The goal of the computer system
is to assist with these activities, and provide security that is more time consuming then paper records.



Many of these storage facilities are small business run by a single owner.  These organizations have
simple but demanding goals.  Providing quality service through fast and accurate records, being able 
to correctly and timely bill customers, and provide fast and simple security is a must for such an
organization.  Many of these business are to small to have their own accounting department, and relies
on external accounting firms.  Being about to export data to a format that can be used by the customers
accounting firm is a requirement of the organization.  
 
\end{document}

