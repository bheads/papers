% Homework #2 
% COSC552 HCI
%
% Byron Heads
% E00062946

\documentclass[12pt]{article}

\title{Homework Set 2 \\
    COSC552 HCI}
\author{ Byron Heads \\
    E00062946 }
\date{\today}

\begin{document}
\maketitle

\section*{A}

\section*{B}

There are many tool to use in designing a system to support users with 
different disabilities.  Before adding these tools you need to have user
controls that enable or disable different tools that depend on the user
that is logged in to the system.  This prevents the user from having to
use the program's options menu to activate or deactivate the different
tools.

To assist a deaf user the interface needs to ensure that any audible signal
has an equivalent visual signal on screen.  This can include flashing 
icons, shaking windows, or in important situation a pop-up window can
be used.  These visual signals will be used with the audible signals for
all users.

Users with manual dexterity problems can be helped with a tool that
increases the size of on screen controls, and fonts.  The makes clicking
buttons, selecting menus, or selecting text easier.  The system can also
be deployed with a touch screen which may make it easier for some users.

The interface can also include tools to assist users with vision 
impairment by giving the user the option to increase the size of the font
and controls.  The interface will use the operating systems set color
scheme.  Most modern operating system include options such as high contrast
colors to help users, the interface will take advantage of this.

\section*{C}

\begin{itemize}

\item Visual consistency.  All menus, windows, and controls should have a
consistent look and feel to them.  The includes across all minor versions
of the interface.

\item Interface should respond quickly.  Any long running or inconsistent 
time operations should be run on a separate thread with an interface control
indicating the progress of the operation.

\item Operations should stoppable.  Operation should be able to be
stopped, and its changes should be rolled back to before the operation
was stopped.

\item Confirm dangerous operations.  Operations that cannot be undone 
should be confirmed with the user before it is done.

\item Do not crowd the screen.  Do not fill the screen with a large number
of controls or information.  This can confuse or overwhelm the user.

\item Do not punish users for mistakes.  Users will make mistakes.  Design
the interface to recover gracefully from these mistakes.  This can include
backups, and saving or allowing users the edit information they have
spent hours to input into the system.

\item Reuse as much as possible.  Reusing interface menus and components
makes the users more comfortable, and easier for them to learn and remember
how to use the interface.

\item Use the users language.  The interface should be in the users
language.  This includes language related to the field or business the
interface is designed for.

\item Design for recognition and not recall.  The user should be able to
recognize how to use the interface, and should not have to remember a set 
of complex steps to perform an operation.

\item Access to help and documentation.  The user should be able to access 
a searchable help system from within the interface.  The user should also
be able to access documentation outside of the interface.  

\item Help users understand and recover from errors.  Errors should be
express in plain and understandable language.  It should indicate the 
problem and give possible solutions.

\end{itemize}

\section*{D}

I am describing my metaphor based on my small business storage management 
and gate control system.

The metaphor used for the storage site is based a 2D map of the facility.
The map shows the location of all storage units, parking spaces, and the 
gate.  Units on the map are color coded to give the manager a quick view
of the status of the facility.  Rented units and spaces would be in blue,
renters that have not paid are in red, unrentable units would be in orange,
and units that are ready to be rented are in green.  Each unit is marked 
with a number and the size of the unit.  This screen also includes options
to change the color coding based on other attributes.  This can include 
unit sizes, how long a renter has had a unit, unit income over time, and
how often a unit is late on payments.

The site map can be useful for many other operations.  If a costumer wants 
to rent a unit the user can quickly see on the map which units are ready
to be rented.  They can even view rentable units by size.  The user can 
then click on a unit and select "Rent out unit" from the pop up menu.  From
my own experience many users forget which unit number they have, but often
can remember where their unit is on site and can find it on a map.  This
can help the user by keeping them from spending time searching for the
customer in the database.  The user can select a unit on the site and from
the menu and select several options.  These options include "Apply a
payment", "Move out", "Contact customer", "Change Rate", or "View details".



\end{document}

